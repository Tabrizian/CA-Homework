%%%%%%%%%%%%%%%%%%%%%%%%%%%%%%%%%%%%%%%%%
% Structured General Purpose Assignment
% LaTeX Template
%
% This template has been downloaded from:
% http://www.latextemplates.com
%
% Original author:
% Ted Pavlic (http://www.tedpavlic.com)
%
% Note:
% The \lipsum[#] commands throughout this template generate dummy text
% to fill the template out. These commands should all be removed when
% writing assignment content.
%
%%%%%%%%%%%%%%%%%%%%%%%%%%%%%%%%%%%%%%%%%

%----------------------------------------------------------------------------------------
%	PACKAGES AND OTHER DOCUMENT CONFIGURATIONS
%----------------------------------------------------------------------------------------

\documentclass{article}

\usepackage{fancyhdr} % Required for custom headers
\usepackage{lastpage} % Required to determine the last page for the footer
\usepackage{extramarks} % Required for headers and footers
\usepackage{graphicx} % Required to insert images
\usepackage{lipsum} % Used for inserting dummy 'Lorem ipsum' text into the template
\usepackage{amsmath}
\usepackage{tikz-qtree-compat}
\usepackage{amssymb}
% Margins
\topmargin=-0.45in
\evensidemargin=0in
\oddsidemargin=0in
\textwidth=6.5in
\textheight=9.0in
\headsep=0.25in

\linespread{1.1} % Line spacing

% Set up the header and footer
\pagestyle{fancy}
\lhead{\hmwkAuthorName} % Top left header
\chead{\hmwkClass\ (\hmwkClassInstructor\ \hmwkClassTime): \hmwkTitle} % Top center header
\rhead{\firstxmark} % Top right header
\lfoot{\lastxmark} % Bottom left footer
\cfoot{} % Bottom center footer
\rfoot{Page\ \thepage\ of\ \pageref{LastPage}} % Bottom right footer
\renewcommand\headrulewidth{0.4pt} % Size of the header rule
\renewcommand\footrulewidth{0.4pt} % Size of the footer rule

\setlength\parindent{0pt} % Removes all indentation from paragraphs

%----------------------------------------------------------------------------------------
%	DOCUMENT STRUCTURE COMMANDS
%	Skip this unless you know what you're doing
%----------------------------------------------------------------------------------------

% Header and footer for when a page split occurs within a problem environment
\newcommand{\enterProblemHeader}[1]{
    \nobreak\extramarks{#1}{#1 continued on next page\ldots}\nobreak
    \nobreak\extramarks{#1 (continued)}{#1 continued on next page\ldots}\nobreak
}

% Header and footer for when a page split occurs between problem environments
\newcommand{\exitProblemHeader}[1]{
    \nobreak\extramarks{#1 (continued)}{#1 continued on next page\ldots}\nobreak
    \nobreak\extramarks{#1}{}\nobreak
}

\setcounter{secnumdepth}{0} % Removes default section numbers
\newcounter{homeworkProblemCounter} % Creates a counter to keep track of the number of problems

\newcommand{\homeworkProblemName}{}
\newenvironment{homeworkProblem}[1][Problem \arabic{homeworkProblemCounter}]{ % Makes a new environment called homeworkProblem which takes 1 argument (custom name) but the default is "Problem #"
    \stepcounter{homeworkProblemCounter} % Increase counter for number of problems
    \renewcommand{\homeworkProblemName}{#1} % Assign \homeworkProblemName the name of the problem
    \section{\homeworkProblemName} % Make a section in the document with the custom problem count
    \enterProblemHeader{\homeworkProblemName} % Header and footer within the environment
}{
    \exitProblemHeader{\homeworkProblemName} % Header and footer after the environment
}

\newcommand{\problemAnswer}[1]{ % Defines the problem answer command with the content as the only argument
    \noindent\framebox[\columnwidth][c]{\begin{minipage}{0.98\columnwidth}#1\end{minipage}} % Makes the box around the problem answer and puts the content inside
}

\newcommand{\homeworkSectionName}{}
\newenvironment{homeworkSection}[1]{ % New environment for sections within homework problems, takes 1 argument - the name of the section
    \renewcommand{\homeworkSectionName}{#1} % Assign \homeworkSectionName to the name of the section from the environment argument
    \subsection{\homeworkSectionName} % Make a subsection with the custom name of the subsection
    \enterProblemHeader{\homeworkProblemName\ [\homeworkSectionName]} % Header and footer within the environment
}{
    \enterProblemHeader{\homeworkProblemName} % Header and footer after the environment
}

%----------------------------------------------------------------------------------------
%	NAME AND CLASS SECTION
%----------------------------------------------------------------------------------------

\newcommand{\hmwkTitle}{Assignment\ \#4} % Assignment title
\newcommand{\hmwkDueDate}{Friday,\ March\ 4,\ 2016} % Due date
\newcommand{\hmwkClass}{Computer Architecture} % Course/class
\newcommand{\hmwkClassTime}{10:45 am} % Class/lecture time
\newcommand{\hmwkClassInstructor}{Dr. Zarandi} % Teacher/lecturer
\newcommand{\hmwkAuthorName}{Iman Tabrizian} % Your name

%----------------------------------------------------------------------------------------
%	TITLE PAGE
%----------------------------------------------------------------------------------------

\title{
    In the name of God\\
    \vspace{2in}
    \textmd{\textbf{\hmwkClass:\ \hmwkTitle}}\\
    \normalsize\vspace{0.1in}\small{Due\ on\ \hmwkDueDate}\\
    \vspace{0.1in}\large{\textit{\hmwkClassInstructor\ \hmwkClassTime}}
    \vspace{3in}
}

\author{\textbf{\hmwkAuthorName}}
\date{} % Insert date here if you want it to appear below your name

%----------------------------------------------------------------------------------------

\begin{document}
\maketitle

%----------------------------------------------------------------------------------------
%	TABLE OF CONTENTS
%----------------------------------------------------------------------------------------

%\setcounter{tocdepth}{1} % Uncomment this line if you don't want subsections listed in the ToC

\newpage
\tableofcontents
\newpage

\begin{homeworkProblem}
    \problemAnswer{
        Computer A:$T_{access}=t_1*h_1+(1-h_1)*(t_1+t_2)=0.98*2+0.02*22=2.4ns$\\
        Computer B:$T_{access}=t_1*h_1+(1-h_1)*(t_1+t_2)=0.90*1.2+0.1*21.2=3.2ns$
    }
\end{homeworkProblem}
\begin{homeworkProblem}
    \begin{homeworkSection}{(a)}
         \problemAnswer{
             \begin{center}
                 \begin{tabular}{|c|c|c|c|c|}
                     \hline
                     0&1&...&63&64\\ \hline
                     x&x         &x         &x         &x\\ \hline
                     x&\checkmark&\checkmark&\checkmark&x\\ \hline
                     x&\checkmark&\checkmark&\checkmark&x\\ \hline
                 \end{tabular}
             \end{center}
             hit rate = 64.61\%
         }
    \end{homeworkSection}
    \begin{homeworkSection}{(b)}
        \problemAnswer{
           With LRU replacement policy you the hit rate will be the same as
           above because with the above policy also the least recently used element
           will be deleted. hit rate = 64.61\%
        }
    \end{homeworkSection}
\end{homeworkProblem}
\begin{homeworkProblem}
    \problemAnswer{
        Memory size = 64*8 word=512 words.
        Block size = 3 bits.
        Cache set index = 2 bits.
        Tag = 9-2-3 = 4 bits.
        Number of bits for addressing a word = 9 bits.
        Word size = 5 bits.
    }
\end{homeworkProblem}
\begin{homeworkProblem}
    \begin{homeworkSection}{(a)}
    \problemAnswer{

        \begin{tabular}{|c|}
            \hline
            3     \\ \hline
            null  \\ \hline
            null  \\ \hline
            miss \\
        \end{tabular}
        \begin{tabular}{|c|}
            \hline
            3  \\ \hline
            2  \\ \hline
            null  \\ \hline
            miss \\
        \end{tabular}
        \begin{tabular}{|c|}
            \hline
            3  \\ \hline
            2  \\ \hline
            1  \\ \hline
            miss \\
        \end{tabular}
        \begin{tabular}{|c|}
            \hline
            0  \\ \hline
            2  \\ \hline
            1  \\ \hline
            miss \\
        \end{tabular}
        \begin{tabular}{|c|}
            \hline
            0  \\ \hline
            3  \\ \hline
            1  \\ \hline
            miss \\
        \end{tabular}
        \begin{tabular}{|c|}
            \hline
            0  \\ \hline
            3  \\ \hline
            2  \\ \hline
            miss \\
        \end{tabular}
        \begin{tabular}{|c|}
            \hline
            4  \\ \hline
            3  \\ \hline
            2  \\ \hline
            miss \\
        \end{tabular}
        \begin{tabular}{|c|}
            \hline
            4  \\ \hline
            3  \\ \hline
            2  \\ \hline
            hit \\
        \end{tabular}
        \begin{tabular}{|c|}
            \hline
            4  \\ \hline
            3  \\ \hline
            2  \\ \hline
            hit \\
        \end{tabular}
        \begin{tabular}{|c|}
            \hline
            4  \\ \hline
            1  \\ \hline
            2  \\ \hline
            miss \\
        \end{tabular}
        \begin{tabular}{|c|}
            \hline
            4  \\ \hline
            1  \\ \hline
            0  \\ \hline
            miss \\
        \end{tabular}
        \begin{tabular}{|c|}
            \hline
            4  \\ \hline
            1  \\ \hline
            0  \\ \hline
            hit \\
        \end{tabular}
        \\
        miss rate=9/12=0.75
    }
\end{homeworkSection}
\begin{homeworkSection}{(b)}
    \problemAnswer{

        \begin{tabular}{|c|}
            \hline
            3     \\ \hline
            null  \\ \hline
            null  \\ \hline
            null  \\ \hline
            miss \\
        \end{tabular}
        \begin{tabular}{|c|}
            \hline
            3  \\ \hline
            2  \\ \hline
            null  \\ \hline
            null  \\ \hline
            miss \\
        \end{tabular}
        \begin{tabular}{|c|}
            \hline
            3  \\ \hline
            2  \\ \hline
            1  \\ \hline
            null  \\ \hline
            miss \\
        \end{tabular}
        \begin{tabular}{|c|}
            \hline
            3  \\ \hline
            2  \\ \hline
            1  \\ \hline
            0  \\ \hline
            miss \\
        \end{tabular}
        \begin{tabular}{|c|}
            \hline
            3  \\ \hline
            2  \\ \hline
            1  \\ \hline
            0  \\ \hline
            hit \\
        \end{tabular}
        \begin{tabular}{|c|}
            \hline
            3  \\ \hline
            2  \\ \hline
            1  \\ \hline
            0  \\ \hline
            hit \\
        \end{tabular}
        \begin{tabular}{|c|}
            \hline
            4  \\ \hline
            2  \\ \hline
            1  \\ \hline
            0  \\ \hline
            miss \\
        \end{tabular}
        \begin{tabular}{|c|}
            \hline
            4  \\ \hline
            3  \\ \hline
            1  \\ \hline
            0  \\ \hline
            miss \\
        \end{tabular}
        \begin{tabular}{|c|}
            \hline
            4  \\ \hline
            3  \\ \hline
            2  \\ \hline
            0  \\ \hline
            miss \\
        \end{tabular}
        \begin{tabular}{|c|}
            \hline
            4  \\ \hline
            3  \\ \hline
            2  \\ \hline
            1  \\ \hline
            miss \\
        \end{tabular}
        \begin{tabular}{|c|}
            \hline
            0  \\ \hline
            3  \\ \hline
            2  \\ \hline
            1  \\ \hline
            miss \\
        \end{tabular}
        \begin{tabular}{|c|}
            \hline
            0  \\ \hline
            4  \\ \hline
            2  \\ \hline
            1  \\ \hline
            miss \\
        \end{tabular}\\
        miss rate = 10 / 12 = 0.833333\\

        The problem is the wrong replacement policy because the data which has
        been added first isn't always the data that must be deleted. The replacement
        policy must be replaced with LRU so that the hit rate increases.
    }
\end{homeworkSection}
\end{homeworkProblem}
\end{document}
